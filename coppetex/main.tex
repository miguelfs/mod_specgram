% !Mode:: "TeX:UTF-8"

\documentclass[tcc,numbers]{coppe}
\usepackage{amsmath,amssymb}
\usepackage{hyperref}

% No need for inputec on LuaLaTeX, just the !Mode above
\usepackage[T1]{fontenc}

% Images path
\graphicspath{{./images/}}

% Make sybols and abbrv pages
\makelosymbols
\makeloabbreviations

\begin{document}
    \title{Espectrogramas de Modulação e Aplicações}
    \foreigntitle{Thesis Title}
    \author{Miguel Fernandes}{de Sousa}
    \advisor{Prof.}{Luiz Wagner}{Pereira Biscainho}{D.Sc.}
    
    \examiner{Prof.}{Nome do Primeiro Examinador Sobrenome}{D.Sc.}
    \examiner{Prof.}{Nome do Segundo Examinador Sobrenome}{Ph.D.}
    \examiner{Prof.}{Nome do Terceiro Examinador Sobrenome}{D.Sc.}
    \department{DEL}
    \date{12}{2021}
    
    \keyword{Primeira palavra-chave}
    \keyword{Segunda palavra-chave}
    \keyword{Terceira palavra-chave}
    
    \maketitle
    
    \frontmatter
    %\makecatalog

    \pagebreak
\pagestyle{plain}
\begin{center}
Declaração de Autoria e de Direitos
\end{center}

\vspace{0.5cm}

Eu,\@author \emph{Matheus Fernandes Moreno} (CPF \emph{169.346.637-64}), autor da monografia \emph{Análise Comparativa de Métodos de Identificação de Sistemas Não-Lineares Aplicados a Sinais de Áudio}, subscrevo para os devidos fins, as seguintes informações:

\begin{enumerate}
    \item O autor declara que o trabalho apresentado na disciplina de Projeto de Graduação da Escola Politécnica da UFRJ é de sua autoria, sendo original em forma e conteúdo.
    \item Excetuam-se do item 1. eventuais transcrições de texto, figuras, tabelas, conceitos e ideias, que identifiquem claramente a fonte original, explicitando as autorizações obtidas dos respectivos proprietários, quando necessárias.
    \item O autor permite que a UFRJ, por um prazo indeterminado, efetue em qualquer mídia de divulgação, a publicação do trabalho acadêmico em sua totalidade, ou em parte. Essa autorização não envolve ônus de qualquer natureza à UFRJ, ou aos seus representantes.
    \item O autor pode, excepcionalmente, encaminhar à Comissão de Projeto de Graduação, a não divulgação do material, por um prazo máximo de 01 (um) ano, improrrogável, a contar da data de defesa, desde que o pedido seja justificado, e solicitado antecipadamente, por escrito, à Congregação da Escola Politécnica.
    \item O autor declara, ainda, ter a capacidade jurídica para a prática do presente ato, assim como ter conhecimento do teor da presente Declaração, estando ciente das sanções e punições legais, no que tange a cópia parcial, ou total, de obra intelectual, o que se configura como violação do direito autoral previsto no Código Penal Brasileiro no art.184 e art.299, bem como na Lei 9.610.
    \item O autor é o único responsável pelo conteúdo apresentado nos trabalhos acadêmicos publicados, não cabendo à UFRJ, aos seus representantes,  ou ao(s) orientador(es), qualquer responsabilização/ indenização nesse sentido.
    \item Por ser verdade, firmo a presente declaração.
\end{enumerate}

\vspace{0.75cm}
\begin{flushright}
\parbox{10cm}{
    \hrulefill

    \vspace{-0.2cm}
    \centering{Matheus Fernandes Moreno}

    \vspace{0.1cm}
}
\end{flushright}
    \pagebreak
\pagestyle{plain}

\noindent
UNIVERSIDADE FEDERAL DO RIO DE JANEIRO\\
Escola Politécnica - Departamento de Eletrônica e de Computação\\
Centro de Tecnologia, bloco H, sala H-217, Cidade Universitária\\
Rio de Janeiro - RJ - CEP 21949-900

\vspace{0.5cm}
Este exemplar é de propriedade da Universidade Federal do Rio de Janeiro, que poderá incluí-lo em base de dados, armazenar em computador, microfilmar ou adotar qualquer forma de arquivamento.

É permitida a menção, reprodução parcial ou integral e a transmissão entre bibliotecas deste trabalho, sem modificação de seu texto, em qualquer meio que esteja ou venha a ser fixado, para pesquisa acadêmica, comentários e citações, desde que sem finalidade comercial e que seja feita a referência bibliográfica completa.

Os conceitos expressos neste trabalho são de responsabilidade do(s) autor(es).
    \chapter*{Agradecimentos}

Gostaria de agradecer a todos.
    \begin{abstract}

Resumo do projeto.

\vspace*{7mm}
\noindent \textit{Palavras-chave:} trabalho, resumo, interesse, projeto final.

\end{abstract}
    \begin{foreignabstract}

Abstract of the project.

\vspace*{7mm}
\noindent \textit{Keywords:} word, word, word.

\end{foreignabstract}
    
    \tableofcontents
    \listoffigures
    \listoftables
    \printloabbreviations
    \printlosymbols
    
    \mainmatter
    \chapter{Introdu{\c c}\~ao}
    
    Segundo a norma de formata{\c c}\~ao de teses e disserta{\c c}\~oes do
    Instituto Alberto Luiz Coimbra de P\'os-gradua{\c c}\~ao e Pesquisa de
    Engenharia (COPPE), toda abreviatura deve ser definida antes de
    utilizada.\abbrev{COPPE}{Instituto Alberto Luiz Coimbra de P\'os-gradua{\c
    c}\~ao e Pesquisa de Engenharia}
    
    Do mesmo modo, \'e imprescind\'ivel definir os s\'imbolos, tal como o
    conjunto dos n\'umeros reais $\mathbb{R}$ e o conjunto vazio $\emptyset$.
    \symbl{$\mathbb{R}$}{Conjunto dos n\'umeros reais}
    \symbl{$\emptyset$}{Conjunto vazio}
    
    \begin{longquote}
    Um exemplo de citação longa nas regras da ABNT (4cm de recuo e fonte menor)
    feita com o ambiente  \verb=longquote= The primary objective of this
    investigation was to determine the feasibility of detecting corrosion in
    aluminum Naval aircraft components with neutron radiographic interrogation
    and the use of standard corrosion penetrameters. Secondary objectives
    included the determination of the effect of object thickness on image quality,
    the defining of minimum levels of detectability and a preliminary investigation
    of a means whereby the degree of corrosion could be quantified with neutron
    radiographic data.
    % \cite{article-example}
    \end{longquote}
    
    % \chapter{Revis\~ao Bibliogr\'afica}
    
    Para ilustrar a completa ades\~ao ao estilo de cita{\c c}\~oes e listagem de
    refer\^encias bibliogr\'aficas, a Tabela apresenta cita{\c
    c}\~oes de alguns dos trabalhos contidos na norma fornecida pela CPGP da
    COPPE, utilizando o estilo num\'erico.

    \backmatter
    \bibliographystyle{coppe-unsrt}
    \bibliography{bibliografia}
    
    \appendix
    \chapter{Algumas Demonstra{\c c}\~oes}
\end{document}