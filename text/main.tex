% !Mode:: "TeX:UTF-8"

\documentclass[tcc,numbers]{coppe}
\usepackage{amsmath,amssymb}
\usepackage{hyperref}

% No need for inputec on LuaLaTeX, just the !Mode above
\usepackage[T1]{fontenc}

% Images path
\graphicspath{{./images/}}

% Make sybols and abbrv pages
\makelosymbols
\makeloabbreviations

\begin{document}
    \title{Espectrogramas de Modulação e Aplicações}
    \foreigntitle{Thesis Title}
    \author{Miguel Fernandes}{de Sousa}
    \advisor{Prof.}{Luiz Wagner}{Pereira Biscainho}{D.Sc.}
    
    \examiner{Prof.}{Nome do Primeiro Examinador Sobrenome}{D.Sc.}
    \examiner{Prof.}{Nome do Segundo Examinador Sobrenome}{Ph.D.}
    \examiner{Prof.}{Nome do Terceiro Examinador Sobrenome}{D.Sc.}
    \department{DEL}
    \date{12}{2021}
    
    \keyword{Primeira palavra-chave}
    \keyword{Segunda palavra-chave}
    \keyword{Terceira palavra-chave}
    
    \maketitle
    
    \frontmatter
    %\makecatalog

    \input{preamble/declaracao}
    \input{preamble/copyright}
    \input{preamble/agradecimentos}
    \begin{abstract}

Resumo do projeto.

\vspace*{7mm}
\noindent \textit{Palavras-chave:} trabalho, resumo, interesse, projeto final.

\end{abstract}
    \begin{foreignabstract}

Abstract of the project.

\vspace*{7mm}
\noindent \textit{Keywords:} word, word, word.

\end{foreignabstract}
    
    \tableofcontents
    \listoffigures
    \listoftables
    \printloabbreviations
    \printlosymbols
    
    \mainmatter
    \chapter{Introdu{\c c}\~ao}
    
    Segundo a norma de formata{\c c}\~ao de teses e disserta{\c c}\~oes do
    Instituto Alberto Luiz Coimbra de P\'os-gradua{\c c}\~ao e Pesquisa de
    Engenharia (COPPE), toda abreviatura deve ser definida antes de
    utilizada.\abbrev{COPPE}{Instituto Alberto Luiz Coimbra de P\'os-gradua{\c
    c}\~ao e Pesquisa de Engenharia}
    
    Do mesmo modo, \'e imprescind\'ivel definir os s\'imbolos, tal como o
    conjunto dos n\'umeros reais $\mathbb{R}$ e o conjunto vazio $\emptyset$.
    \symbl{$\mathbb{R}$}{Conjunto dos n\'umeros reais}
    \symbl{$\emptyset$}{Conjunto vazio}
    
    \begin{longquote}
    Um exemplo de citação longa nas regras da ABNT (4cm de recuo e fonte menor)
    feita com o ambiente  \verb=longquote= The primary objective of this
    investigation was to determine the feasibility of detecting corrosion in
    aluminum Naval aircraft components with neutron radiographic interrogation
    and the use of standard corrosion penetrameters. Secondary objectives
    included the determination of the effect of object thickness on image quality,
    the defining of minimum levels of detectability and a preliminary investigation
    of a means whereby the degree of corrosion could be quantified with neutron
    radiographic data.
    % \cite{article-example}
    \end{longquote}
    
    % \chapter{Revis\~ao Bibliogr\'afica}
    
    Para ilustrar a completa ades\~ao ao estilo de cita{\c c}\~oes e listagem de
    refer\^encias bibliogr\'aficas, a Tabela apresenta cita{\c
    c}\~oes de alguns dos trabalhos contidos na norma fornecida pela CPGP da
    COPPE, utilizando o estilo num\'erico.

    \backmatter
    \bibliographystyle{coppe-unsrt}
    \bibliography{bibliografia}
    
    \appendix
    \chapter{Algumas Demonstra{\c c}\~oes}
\end{document}